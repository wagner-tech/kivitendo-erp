\documentclass{scrartcl}
\usepackage[utf8]{inputenc}% muss zum Editor passen -> http://texwelt.de/wissen/fragen/2656/
\usepackage[T1]{fontenc}
\usepackage[ngerman]{babel}
\usepackage{multicol}
\usepackage{wallpaper}
\usepackage{hyperref} 

\renewcommand{\familydefault}{\sfdefault}

\title{kivitendo: Ein Programm für Buchhaltung und Lagerwesen}
\author{WagnerTech UG, Turfstr. 18a, 81929 München, www.wagnertech.de}
\date{\vspace{-5ex}}

\ULCornerWallPaper{1}{wagner_tech_briefbogen_blau_fs1.pdf}

\begin{document}

\maketitle

\begin{abstract}
Kivitendo zur Buchhaltung: Kivitendo ist ein browserbasiertes Buchhaltungsprogramm, mehrbenutzer- und mandantenfähig (\url{http://www.kivitendo.org}).

\end{abstract}

\begin{multicols}{2}
Im Bereich der OpenSource-Software gibt es heute viele Programme, die für die geschäftlichen Anforderungen kleiner und mittelgroßer Unternehmen geeignet sind. Der Vorteil von quelloffener Software liegt dabei nicht primär in der kostenlosen Verfügbarkeit, sondern darin, dass sie von jedermann, also auch von uns, für Ihre speziellen Bedürfnisse angepasst werden dürfen.

Das Programm kivitendo ist eine vielseitige, ausgereifte Anwendung, die kleinen und mittelgroßen Firmen
die für die Buchhaltung und das Lagerwesen nötigen Funktionen zur Verfügung stellt. kivitendo ist „das Eine“ für „das Alles“: Basis-ERP-System mit Artikeln, Waren, Kunden, Lieferanten, Lager, Angebote, Rechnungen, Mahnwesen, Buchungsvorbereitung für die Fibu, Kontakte mit grundlegenden CRM-Funktionen und noch einiges mehr. 

WagnerTech UG betreut Sie bei Installation und Einführung und kann das Programm auch nach Ihren speziellen Anforderungen
ergänzen. Eine erste Ergänzung, die wir kivitendo hinzugefügt haben, ist eine Finanzamtsschnittstelle zur Übertragung der
monatlichen Umsatzsteueranmeldung.
\end{multicols}

\end{document}
